\documentclass[journal]{vgtc}                % final (journal style)
%\documentclass[review,journal]{vgtc}         % review (journal style)
%\documentclass[widereview]{vgtc}             % wide-spaced review
%\documentclass[preprint,journal]{vgtc}       % preprint (journal style)

%% Uncomment one of the lines above depending on where your paper is
%% in the conference process. ``review'' and ``widereview'' are for review
%% submission, ``preprint'' is for pre-publication, and the final version
%% doesn't use a specific qualifier.

%% These few lines make a distinction between latex and pdflatex calls and they
%% bring in essential packages for graphics and font handling.
%% Note that due to the \DeclareGraphicsExtensions{} call it is no longer necessary
%% to provide the the path and extension of a graphics file:
%% \includegraphics{diamondrule} is completely sufficient.
%%
\ifpdf%                                % if we use pdflatex
  \pdfoutput=1\relax                   % create PDFs from pdfLaTeX
  \pdfcompresslevel=9                  % PDF Compression
  \pdfoptionpdfminorversion=7          % create PDF 1.7
  \ExecuteOptions{pdftex}
  \usepackage{graphicx}                % allow us to embed graphics files
  \DeclareGraphicsExtensions{.pdf,.png,.jpg,.jpeg} % for pdflatex we expect .pdf, .png, or .jpg files
\else%                                 % else we use pure latex
  \ExecuteOptions{dvips}
  \usepackage{graphicx}                % allow us to embed graphics files
  \DeclareGraphicsExtensions{.eps}     % for pure latex we expect eps files
\fi%

%% it is recomended to use ``\autoref{sec:bla}'' instead of ``Fig.~\ref{sec:bla}''
\graphicspath{{figures/}{pictures/}{images/}{./}} % where to search for the images

\usepackage{microtype}                 % use micro-typography (slightly more compact, better to read)
\PassOptionsToPackage{warn}{textcomp}  % to address font issues with \textrightarrow
\usepackage{textcomp}                  % use better special symbols
\usepackage{mathptmx}                  % use matching math font
\usepackage{times}                     % we use Times as the main font
\renewcommand*\ttdefault{txtt}         % a nicer typewriter font
\usepackage{cite}

% added by Connor
\usepackage{subfig}
\usepackage{balance}

%% If you are submitting a paper to a conference for review with a double
%% blind reviewing process, please replace the value ``0'' below with your
%% OnlineID. Otherwise, you may safely leave it at ``0''.
\onlineid{0}

%% declare the category of your paper, only shown in review mode
\vgtccategory{Research}

%% Paper title - 1 pt for descriptive title
\title{Analysis of Execution Trace Lines in Gantt Charts}

%% This is how authors are specified in the journal style

%% indicate IEEE Member or Student Member in form indicated below
%% 1 pt for name
\author{Connor Scully-Allison}
\authorfooter{
%% insert punctuation at end of each item
\item
 Connor Scully-Allison is a graduate student at the University of Arizona. E-mail: cscully-allison@email.arizona.edu.
}

%other entries to be set up for journal
%\shortauthortitle{Firstauthor \MakeLowercase{\textit{et al.}}: Paper Title}

%% Abstract section - 5 pts
\abstract{
In the HPC domain, a problem that plagues execution trace visualizations is an inability to meaningfully render communications on Gantt charts for modern supercomputers. The number of concurrent processors being monitored makes it impossible to draw all communication connections at once. This is problematic as understanding communication between processes is crucial to optimizing programs which use these HPC resources and must be as performant as possible. To better understand this problem we propose a within-subjects study designed to evaluate what repeated marks may be necessary and which may be redundant in displaying commonly found HPC communication patterns. This experiment will be piloted and iterated on until it is ready to be deployed to a large base of users so that meaningful data can be collected. Data collected from this experiment can help inform a design study which will produce a better Gantt chart. One that can represent communications data at scale.

} % end of abstract

%% Keywords that describe your work. Will show as 'Index Terms' in journal
%% please capitalize first letter and insert punctuation after last keyword
%\keywords{Radiosity, global illumination, constant time}

%% ACM Computing Classification System (CCS). 
%% See <http://www.acm.org/class/1998/> for details.
%% The ``\CCScat'' command takes four arguments.

%\CCScatlist{ % not used in journal version
% \CCScat{K.6.1}{Management of Computing and Information Systems}%
%{Project and People Management}{Life Cycle};
% \CCScat{K.7.m}{The Computing Profession}{Miscellaneous}{Ethics}
%}

%% Uncomment below to include a teaser figure.
%   \teaser{
%   \centering
%   \includegraphics[width=16cm]{CypressView}
%   \caption{In the Clouds: Vancouver from Cypress Mountain.}
%  }

%% Uncomment below to disable the manuscript note
%\renewcommand{\manuscriptnotetxt}{}

%% Copyright space is enabled by default as required by guidelines.
%% It is disabled by the 'review' option or via the following command:
% \nocopyrightspace

\vgtcinsertpkg

%%%%%%%%%%%%%%%%%%%%%%%%%%%%%%%%%%%%%%%%%%%%%%%%%%%%%%%%%%%%%%%%
%%%%%%%%%%%%%%%%%%%%%% START OF THE PAPER %%%%%%%%%%%%%%%%%%%%%%
%%%%%%%%%%%%%%%%%%%%%%%%%%%%%%%%%%%%%%%%%%%%%%%%%%%%%%%%%%%%%%%%%

\begin{document}

%% The ``\maketitle'' command must be the first command after the
%% ``\begin{document}'' command. It prepares and prints the title block.

%% the only exception to this rule is the \firstsection command
\firstsection{Introduction} % or "Motivation"

\maketitle

% Introduction and/or Motivation - 15 pts
For decades, Gantt charts have been used by High Performance Computing (HPC) professionals to profile and analyze the execution of highly-parallel programs running on multiple cores. When profiling, developers are looking for "bottlenecks" in code, or places where parallelism is not as effective as it could be. A common cause of bottlenecks is the useless work being done transferring data or waiting on the execution of processes: "overhead." Gantt charts, showing the parallel execution of many processes over time, are very good for rendering communications between processes with lines showing how long communication took, when it began and where it ends. This notation gives developers a clear topology of how their program is communicating and guides them to opportunities for optimization.

Unfortunately, this reliance on Gantt charts and lines to denote communication poses a unique problem. Where once nearly half of supercomputers in the top 500 of the world had 16 or fewer cores \cite{top500}, a naive approach to developing these visualizations worked well. In Figure \ref{fig:simple_gantt}, we see an example of such a visualization. As supercomputers have grown to thousands and tens-of-thousands of cores however, this approach of visualizing execution traces is no longer effective.

As we can see in Figure \ref{fig:bad_gantt} significant meaning is lost from this chart when it is scaled up into the 100s of cores. Most importantly, it is impossible to see the individual lines connecting calls between processes. They are lost entirely in a sea of visual noise. Even though this software provides functionality to zoom in and explore subsections in greater detail, it is of little value if one cannot determine the key areas to examine. The visualization no longer provides an effective high level view which can be used to quickly identify key message passing patterns and potential bottlenecks.


\begin{figure}[h]
    \centering
    \includegraphics[width=.4\textwidth]{figs/basic_gantt.png}
    \caption{A Gantt chart produced by Vampir from \cite{isaacs2014state}. Functions invoked with the MPI library are shown in red and other executions are indicated with green. The y axes organizes bars by process and the x axis shows execution states of each process over time. Black lines connecting bars show communication between processes via passes messages.}
    \label{fig:simple_gantt}
\end{figure}

What this example shows is that Gantt charts do not scale to meet contemporary needs in HPC. Critical information provided by patterns of message passing lines cannot be rendered at this resolution. This failure provokes the question which drives the subject of this paper: How can we render all the necessary and sufficient communications data required to find potential problems in massively parallel HPC programs within the physical limitations posed by dense charts? What marks are redundant? What marks are necessary? If we can answer these questions, this work will provide a firm foundation upon which to design a more effective visualization for HPC profiling. This in turn, will positively impact the productivity of parallel programmers seeking to optimize critical software.

\begin{figure}[h]
    \centering
    \includegraphics[width=.53\textwidth]{figs/bad_gantt.png}
    \caption{A Gantt chart produced by Vampir from \cite{Brendel2016}. Lines intended to show communication between processes are too numerous to be parsed meaningfully. Furthermore these lines obscure other parts of the chart.}
    \label{fig:bad_gantt}
\end{figure}

In the domain of HPC visualization, and with communication lines on Gantt charts especially, the appropriate balance of necessary and sufficient information is not clearly defined. To address this gap, we propose an empirical study examining how effectively people recognize patterns of lines which simulate message-call patterns in real Gantt charts. This experiment will be implemented as a within-subjects study, evaluating how error rates of pattern recognition are affected by how much of an overall pattern is provided. Does a partial representation of a pattern provide enough meaningful information? Is a full representation too much? What is the threshold between the two? Does the pattern itself impact what marks are redundant or necessary?

This study can provide valuable granular information about what characteristics users identify or lock-on to when processing visual patterns. It can inform possible designs by showing when users can no longer recognize meaningful patterns or by showing how little a user requires for identification and recognition. 

To summarize, the aims of this research are:
\begin{enumerate}
    \item Develop and execute a visualization user study 
    \item Evaluate the results of said user study and draw meaningful conclusions about:
    \begin{enumerate}
        \item People's ability to recognize patterns of the variety commonly found in the lines denoting message passing in line charts
        \item Glean what information is redundant in communication visualizations without suffering significant loss of meaning.
    \end{enumerate}
\end{enumerate}

 

% 10 pts
\section{Background}
\label{sec:background}

This research applies significantly to high performance computing (HPC), so to understand it some understanding of HPC concepts are necessary. First is the concept of profiling. Profiling generally describes the process of measuring a program's execution using various metrics like runtime and memory usage. This information is used to optimize a program and is crucial to understanding the complex and large programs developed to run on distributed systems and supercomputers.

When profiling parallel programs there are many characteristics which experts examine. Among them, communication between processes is one of the most important. This communication is often abstracted as "messages" passed between programs conveying state information or containing data. Because supercomputers commonly do not share memory this data must be transferred between CPU's using physical buses or network connections. The most common library used for this process is called "Message Passing Interface" or MPI. The time taken to transfer this data is called "overhead" and describes a common phenomenon that causes slowdown with parallel programs.


A Gantt chart describes a special variety of chart which plots tasks along a timeline with bars. The x axis commonly denotes time in these charts and the y axis usually denotes a category. In HPC, each bar denotes a particular process or processor running a portion of parallel code. This bar can be subdivided into sections that denote function calls or execution states. Straight lines indicate passed messages between processes and connect bars at various points. An example of a typical Gantt chart can be seen in Figure \ref{fig:simple_gantt}.


% 5 pts
\subsection{Related Work}
\label{sec:related}
Gantt charts have been used as a key visualization method in HPC profiling software for many years. Introduced as a "timeline" view by VAMPIR in 1996, many other platforms adapted the design they presented to visualize execution traces \cite{nagel1996vampir}.  Notably, Intel provides a "Intel Trace Analyzer" software which uses a nearly identical layout to VAMPIR\cite{Intel2019}. HPCToolkit similarly provides a Gantt chart view tracing the execution of various processes over time, however they omit the lines denoting message passing between processes\cite{Adhianto2009}.

As is implied by HPCToolkit, the use of connecting lines to denote passed messages is not universal. However it can provide critical contextual information which aids researchers looking for bottlenecks. This approach was notably introduced as a means to give programmers insight into the structure of communication between threads and cores\cite{Leblanc1990}. At small scales this insight can be highly informative as to how communication overhead may impact the efficiency of parallel programs. 

The concept of developing a Gantt chart for the modern era of HPC is not a new one. Acknowledging the failure of traditional techniques for HPC profiling and visualization, there has been continual effort to improve the general layout of Gantt charts to provide more meaningful information \cite{Cottam2015, Adhianto2016HPCToolkit, Tallent2011}. Although impactful, much of this work is generally focused on an improvement of scaling process bars in timelines and coding more information into those bars. 

More specific to the subject of this paper, there does exist some work which attempts to preserve the message passing information encoded with lines at modern scales. Researchers in Germany have attempted integrate a commonly found graph visualization technique called "edge bundling" into Gantt charts to reduce visual clutter of these connecting lines \cite{Brendel2016}. There has also been research which seeks to mitigate noise caused by unnecessary edge crossings in physical timelines by converting them to logical time\cite{isaacs2014combing}. These approaches have had success in simplifying visualizations but can still obscure other information in their charts and suffer from scaling issues as more and more processes are added. 

Despite being widely used in HPC profiling, there exists very little work evaluating Gantt charts' effectiveness with empirical, user-study based research. The closest work in recent years, which provides a good model for this sort of evaluation comes from Sambasivan et al \cite{Sambasivan2013}. This work on request flow comparisons provides us with some insight into how best to design a study in this domain and some metrics to consider for evaluation.



\section{Research Plan} 
\label{sec:research}

% 30 pts for explanation here combined with timeline below

Having identified that we lack understanding of what information is redundant or necessary to meaningfully represent message passing with high volume Gantt charts, we must evaluate the visual notation we use to indicate message passing. To evaluate this notation we will perform a controlled experiment to evaluate how people perceive patterns and gauge the limitations of perception of those patterns. In order to generalize conclusions and mitigate the need for domain experts as subjects, this study will abstract away the literal use of a Gantt chart and represent patterns as lines connecting columns of rectangles.

\begin{figure}[h]
    \centering
    \includegraphics[width=.4\textwidth]{figs/study.png}
    \caption{The current design of the experiment software subjects will interact with. Users are presented with two images of connected columns and asked to compare the patterns of lines within each.}
    \label{fig:study}
\end{figure}


The intended experiment will be a within-subjects experiment with two independent variables comprised of two conditions and 4 conditions respectively. Users will be asked to compare a provided visualization of lines connecting boxes and compare it against another similar visualization. The first independent variable will be how much information we provide users to compare with: a partial rendering of the pattern or full rendering. Users will respond yes or no as to whether the provided pattern matches with a comparator we will swap out at random. The second independant variable will be the pattern itself, which could be one of four: a basic offset, a ring, a stencil, and an exchange. Examples of these patterns can be seen in Figure \ref{fig:patterns}. The dependant variable we measure from these responses is error rate -- how successfully users compared patterns in the face of. An example of the experiment layout can be seen in Figure \ref{fig:study}.


\begin{figure*}[t!]
     \subfloat[An example of an offset pattern.\label{fig:offset}]{%
       \includegraphics[width=0.2\textwidth, height=0.3\textwidth]{figs/offset_normal.png}
     }
     \hfill
     \subfloat[An example of an ring pattern. \label{fig:ring}]{%
       \includegraphics[width=0.2\textwidth, height=0.3\textwidth]{figs/ring_normal.png}
     }
    \hfill
    \subfloat[An example of a stencil pattern.\label{fig:stencil}]{%
       \includegraphics[width=0.2\textwidth, height=0.3\textwidth]{figs/trace_normalized.png}
     }
     \hfill
     \subfloat[An example of a exchange pattern.\label{fig:exchange}]{%
       \includegraphics[width=0.2\textwidth, height=0.3\textwidth]{figs/exchange_normal.png}
     }
    \caption{Examples of various execution trace patterns.}
    \label{fig:patterns}
\end{figure*}

By measuring error rate, we will draw a connection between how much context a user is provided and if it is too much or too little to make meaningful conclusions. The experiment itself will be piloted with students from the University of Arizona who will also be asked to vocalize their inner thought processes so that we can better understand how the dependant variable is influenced by changes to the independent variable. From this pilot study we will to iterate on the design of this experiment to address weak points. From there, we will bring this study to the Supercomputing conference and collect data from the large population of attendees.

This large volume of collected data will be transformed and analyzed for validity using a statistical test. The derived error rates, being ratio data, will be analyzed with an ANOVA analysis to prove that experiment results are statistically significant. From here we will derive conclusions on what specific characteristics lead to higher or lower error rates by respondents, if possible.


% 5 pts
\subsection{Data}
\label{sec:data}

The data being used for this experiment is manufactured to produce the visualizations which subjects will be asked to compare. It is JSON encoded descriptions of the visual structure of question charts and corresponding comparison charts. This data is parsed and visualized by the experiment software. Some question/answer pairs have been provided by the project PI, however more will be produced as part of this experiment. 

% 15 pts
\subsection{Evaluation}
\label{sec:eval}
As alluded to above in Section \ref{sec:research}, this experiment will be evaluated with formal hypothesis testing. Hypothesis testing describes the process of performing statistical tests on experiment data to disprove the \textit{null hypothesis}. This hypothesis states that the independent variable(s) had no affect on the dependant variable(s) and that the results were completely random. By disproving this hypothesis we conclude that swapping out conditions in our independent variables did have some measurable effect on how subjects behaved. Anticipating that our error rates are ratio data, and expecting a normal distribution on that data, we plan to use an ANOVA analysis for this hypothesis testing. This plan could change as data is collected and a different test is deemed more appropriate. 

On a longer timescale, we expect to integrate the conclusions drawn from this study into a larger design study on Gantt charts. By building a visualization using ideas developed from this study we can validate those conclusions with the effectiveness of newly designed charts. And the effectiveness of these charts, can | in turn | be validated with various metrics such as community adoption rates, formal usability experiments, and comparative performance evaluations. The success of a new visualization or representation of communications in these charts would provide significant validation for the work proposed in this document.


% 5 pts
\subsection{Technology}
\label{sec:tech}

The core experiment platform which runs the experiment is a web application primarily coded in Javascript, HTML and CSS. D3js is used to execute the programmatic drawing of boxes and connecting lines. Data management functionality, like collection of user responses, saving and processing data is managed on the server side with python scripts. The \textit{flask} python library is used for serving down web pages and communicating with the client. These technologies are very commonly used in client server architectures and provide a robust means to quickly develop and iterate on an experiment platform which has a limited range of user interaction and minimal performance needs.

\subsection{Timeline}
\label{sec:timeline}

\vspace{1.5ex}\noindent\textbf{Project Milestone Two} 
For the second project milestone, experimental methodology will be expanded on and explained in greater detail. Any changes to the proposed independent and dependent variables will be noted with clear explanations of what provoked these changes and what better information can be gleaned from the experiment with these changes. Current development on the experiment platform should be complete enabling the execution of a pilot study on location at the University of Arizona. Full descriptions of the experiment platform and how it works will be included in the report at this milestone. Related works and background will be updated to reflect the author's research and expanded understanding of this domain.

\vspace{1.5ex}\noindent\textbf{Project Milestone Three}
By this project milestone, data will have been collected from the pilot study executed after milestone two. This data will be analyzed and presented alongside derived conclusions in this report. Planned changes to the experiment design resulting from this pilot experiment  will highlighted here. Changes to the experiment platform will also be detailed in this report and will be justified in the context of the pilot study. The experiment platform should be containerized at this juncture using Docker or some other technology and hosted on a server for general access. 

\vspace{1.5ex}\noindent\textbf{Project Milestone Four} 
By this milestone, the changes to the experiment design and experiment platform will be fully integrated into the project. A working and robust version of this software will be made available at the Supercomputing conference and a large scale experiment will be executed. Expectations and hypotheses will be outlined in the corresponding report and project progress will be detailed here.


\vspace{1.5ex}\noindent\textbf{Project Milestone Five} 
At this project milestone, data collected from a attendees to the Supercomputing conference will be aggregated, aligned, visualized and tested for statistical significance. Conclusions will be drawn from this data and reported on. Depending on the results of this experiment, the next steps will be clearly outlined in this report whether it is another iteration on this experiment's design or the application of derived conclusions.





\begin{table}[h]
%% Table captions on top in journal version
 \caption{Project Milestones}\vspace{1ex} % the \vspace adds some space after the top caption
 \label{tab:milestones}
 \scriptsize
 \centering % avoid the use of \begin{center}...\end{center} and use \centering instead (more compact)
   \begin{tabular}{l|p{.75\linewidth}}
     Milestone & Description (\%)\\
   \hline
     PM2 & Current development on the experiment platform complete, code available in online repository, experiment design more clearly articulated\\
     PM3 & Pilot study executed, data analyzed and conclusions drawn, proposals of adjustments to study design and software, software hosted on a server\\
     PM4 & Changes to study design and software fully implemented, initial data collected from large study, pre-experiment expectations and hypotheses clearly outlined\\
     PM5 & data analysis on data collected in milestone 4, conclusions drawn, clear plan on this study's iteration or adaptation of lessons learned\\
   \end{tabular}
\end{table}



% 10 pts
\section{Impacts}
\label{sec:impact}

The primary impact of this research is quantifying how people derive meaningful information from marks of lines and patterns of lines, specifically in the context of tracing communication in parallel programs. Even if the data obtained from this experiment supports common sense expectations, this work can then be used to justify design decisions made by future iterations of existing Gantt chart visualization software.

More concretely, the conclusions drawn from this work can be leveraged to inform a guided design study with the goal of building a better Gantt chart that effectively visualizes communication at large scales. Innovation on this front can significantly impact contemporary profiling workflows with large, distributed supercomputer clusters. If future profiling visualization software can show these communication patterns using more succinct notation, domain professionals will be able to locate bottlenecks significantly faster than current software allows for.



%\bibliographystyle{abbrv}
\bibliographystyle{abbrv-doi-hyperref}
%%use following if all content of bibtex file should be shown
%\nocite{*}
\balance
\bibliography{proposal}
\end{document}

